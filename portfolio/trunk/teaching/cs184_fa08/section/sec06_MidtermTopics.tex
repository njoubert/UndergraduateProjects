%
%  untitled
%
%  Created by Niels Joubert on 2008-10-06.
%  Copyright (c) 2008 __MyCompanyName__. All rights reserved.
%
\documentclass[8pt]{article}

% Use utf-8 encoding for foreign characters
\usepackage[utf8]{inputenc}

% Setup for fullpage use
\usepackage{fullpage}
%\addtolength{\topmargin}{+.0in}
% \addtolength{\topmargin}{+.0in}
\addtolength{\oddsidemargin}{-.5in}
\addtolength{\evensidemargin}{-.5in}
\addtolength{\textwidth}{1in}

\addtolength{\topmargin}{-.5in}
\addtolength{\textheight}{1.75in}



% Uncomment some of the following if you use the features
%
% Running Headers and footers
%\usepackage{fancyhdr}

% Multipart figures
%\usepackage{subfigure}

% More symbols
\usepackage{amsthm}
\usepackage{amsfonts}
\usepackage{amsmath}
\usepackage{amssymb}
\usepackage{latexsym}

% Surround parts of graphics with box
\usepackage{boxedminipage}

% Package for including code in the document
\usepackage{listings}
\usepackage{verbatim}

% If you want to generate a toc for each chapter (use with book)
\usepackage{minitoc}

% This is now the recommended way for checking for PDFLaTeX:
\usepackage{ifpdf}

%\newif\ifpdf
%\ifx\pdfoutput\undefined
%\pdffalse % we are not running PDFLaTeX
%\else
%\pdfoutput=1 % we are running PDFLaTeX
%\pdftrue
%\fi

\ifpdf
\usepackage[pdftex]{graphicx}
\else
\usepackage{graphicx}
\fi
\title{Midterm Topics List}
\author{ Niels Joubert }

\date{2008-10-07}

\begin{document}

\ifpdf
\DeclareGraphicsExtensions{.pdf, .jpg, .tif}
\else
\DeclareGraphicsExtensions{.eps, .jpg}
\fi

\maketitle

\begin{itemize}
    
    \item \textbf{Color}
    
    \begin{itemize}
        
        \item Eye's mechanism and perception of color
        
        \item Full spectra and our RGB color model
        
        \item Additive and Subtractive mixing
        
        \item Chromacity and the CIE color space
        
        \item Color Phenomenon - reflection, transmission, scattering, etc.
        
    \end{itemize}
    
    \item \textbf{Shading}

    \begin{itemize}
        
        \item Local Shading Phenomenon - what is, what isn't
        
        \item Pointwise Shading
        
        \begin{itemize}
        
            \item BRDF - Full, Component-wise and Extended
        
            \item Approximate BRDF as used in our shading model (section 3 handout)
        
            \item Lambertian (Diffuse) materials
        
        \end{itemize}
        
        \item Surface shading - Phong, Gouraud and Flat Shading
        
    \end{itemize}
    
    \item \textbf{Linear Transformations}
    
    \begin{itemize}
        
        \item 2D - Scale, Rotate, Shear, Translate
        
        \item 3D - Scale, Shear, Translate
        
        \item Homogenized Coordinates (2D and 3D)
        
        \item 3D Rotations
        
        \begin{itemize}
            
            \item Axis-Aligned Rotations
            
            \item Euler Angles
            
            \item Exponential Maps
            
            \item Quaternions
            
        \end{itemize}
        
    \end{itemize}
    
    \item \textbf{Perspective Transforms} (yes, perspective transforms often are linear transforms)
    
    \begin{itemize}
     
        \item What is it? It is present in Raytracing? Scan Conversion?
        
        \item Linear Projection
        
        \item Orthogonal Projection
        
        \item Canonical View Space - ``placing your eye at an arbitrary position and direction''
        
        \item Perspective Projection
        
        \begin{itemize}
            \item Theory
            \item Vanishing Points
            \item Visual Tricks
            \item Depth Distortion
        \end{itemize}
       
    \end{itemize}
    
    \item \textbf{Raytracing}
    
    \begin{itemize}
     
        \item Why Raytracing? What can we model with it, what can't we? What is expensive, what is cheap?
     
        \item General Algorithm
        
        \begin{itemize}
            \item Different ``kinds'' of rays - shadow, reflection, etc - and how we incorporate them.
            \item Intersection tests - sphere and triangle.
        \end{itemize}
        
        \item Extra additions - soft shadows, area lights, motion blur, DoF, etc.
        
        \item Speedup Structures and BSPs - Know how to build and traverse BSPs on paper!
       
    \end{itemize}
    
    
\end{itemize}

\end{document}
